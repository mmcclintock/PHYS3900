% latex template for a report
% written by Michael McClintock

\RequirePackage[l2tabu, orthodox]{nag} % warnings for obsolete stuff
\documentclass[12pt,a4paper]{article}  % article class

% font setup
\usepackage{ifluatex}
\ifluatex
    \usepackage[no-math]{fontspec} % keep standard math
    \setmainfont[Ligatures=TeX]{Linux Libertine}
    \setsansfont[Ligatures=TeX]{Linux Biolinum}
    \setmonofont[Ligatures=TeX]{Inconsolata}
\else
    \usepackage[T1]{fontenc}
    \usepackage{inconsolata}
\fi
\usepackage{microtype}

% useful packages
\usepackage[svgnames]{xcolor}          % color commands
\usepackage{graphicx}                  % including graphics (use eps)
\usepackage[margin=20mm]{geometry}     % set margins
\usepackage{amsmath,amssymb,cancel,bm} % math \bm for bold math
\usepackage[hidelinks]{hyperref}       % clickable links
\usepackage{url}                       % format urls
\usepackage{cleveref}                  % convenient refrencing use \Cref
\usepackage{paralist}                  % compact itemize/enumerate
\usepackage[hang, small, bf, margin=20pt]{caption} % captions for figures
\usepackage{setspace}%\onehalfspacing               % 1.5 line spacing
\setlength{\parindent}{0cm}\usepackage{parskip}    % paragraph skip
\usepackage[                                   % SI units
    load-configurations=abbreviations,         % unit abrev.
    separate-uncertainty=true,                 % plus minus in uncertainty
    inter-unit-product=\ensuremath{{}\cdot{}}, % dot between units
    per-mode=symbol-or-fraction                % per style
]{siunitx}
% \usepackage{todonotes} % use \todo{blah blah} options inline/noline/nolist/color
% \usepackage{minted}  % code highlighting requires pygments
% \usepackage[english]{babel} % other languages
% \usepackage[nottoc,numbib]{tocbibind} % refs in toc

\begin{document}

\begin{center}
  \Large Peer Review: Addressing Learning Objectives in PHYS1002
  \\[1cm]
\end{center}

As the proposal states the main goals of the project is to highlight
difficulties with the current approach of teaching PHYS1002
(specifically multidisciplinary aspects) and to provide some sample
material that could be used to improve the class. First off, I think
this is an interesting problem. It's the type of problem that is often
overlooked because of various reasons but has the potential to greatly
improve the class.  I like the fact that the proposal is very clear
and succinct. The language is appropriate and very readable. The aim
section is good.

From my point of view the major weakness of the proposal is the
absence of intent to validate the findings objectively. Whilst I agree
that most of the findings (from interviewing students/tutors) will be
subjective and qualitative it should be a goal to find a way to
objectively test the new material that is created for PHYS1002. The
findings will have far more impact if there is some real evidence (not
subjective) that the new material can improve PHYS1002.

In my opinion there are also some minor weaknesses in the proposal.
The section on project significance doesn't actually address any
features of \emph{this} project instead it contains a general
statement and a reference. There should be a reference list at the
end. Finally I feel the expected outcomes sections could be better.
For example, a list of issues in relation to PHYS1002 as found from
interviews should be an expected outcome.

Overall I would give this proposal 5 out of 7.

\end{document}

% v: set spell:
