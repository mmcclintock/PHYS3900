% latex template for a article
% written by Michael McClintock

\RequirePackage[l2tabu, orthodox]{nag} % warnings for obsolete stuff
\documentclass[10pt, a4paper]{article}  % article class

% font setup
\usepackage{ifluatex}
\ifluatex
    \usepackage[no-math]{fontspec} % keep standard math
    \setmainfont[Ligatures=TeX]{Linux Libertine}
    \setsansfont[Ligatures=TeX]{Linux Libertine}
    \setmonofont[Ligatures=TeX]{Inconsolata}
\else
    \usepackage[T1]{fontenc}
    \usepackage{inconsolata}
\fi
\usepackage{microtype}

% useful packages
\usepackage[svgnames]{xcolor}          % color commands
\usepackage{graphicx}                  % including graphics (use eps)
\usepackage{multicol}
\usepackage[margin=15mm]{geometry}     % set margins
\usepackage{amsmath,amssymb,cancel,bm} % math \bm for bold math
\usepackage[hidelinks]{hyperref}       % clickable links
\usepackage{url}                       % format urls
\usepackage{cleveref}                  % convenient refrencing use \Cref
\usepackage{paralist}                  % compact itemize/enumerate
\usepackage[hang, small, bf, margin=20pt]{caption} % captions for figures
\usepackage{setspace}%\onehalfspacing               % 1.5 line spacing
\setlength{\parindent}{0cm}\usepackage{parskip}    % paragraph skip
\usepackage[                                   % SI units
    load-configurations=abbreviations,         % unit abrev.
    separate-uncertainty=true,                 % plus minus in uncertainty
    inter-unit-product=\ensuremath{{}\cdot{}}, % dot between units
    per-mode=symbol-or-fraction                % per style
]{siunitx}
% \usepackage{todonotes} % use \todo{blah blah} options inline/noline/nolist/color
% \usepackage{minted}  % code highlighting requires pygments
% \usepackage[english]{babel} % other languages
% \usepackage[nottoc,numbib]{tocbibind} % refs in toc

\begin{document}

\begin{center}
    \rule[0.5ex]{1\columnwidth}{1pt}
    \\[4mm]
    \textbf{\textsc{\Huge Digital Notebooks for Computation}}
    \\[6mm]
    \textit{\Large A classy interface for scientific computing}
    \\[6mm]
    \textsc{\large By Michael McClintock}
    \\[4mm]
    \rule[0.5ex]{1\columnwidth}{1pt}
\end{center}

\begin{multicols}{2}

Some people are really good at solving problems with pen and paper. There is
just something special about solving a problem with just your mind, a pen and
some paper. 

When you're done there is a feeling of value associated with your pages of
notes and derivations. If you write it clearly with comments well all the
better, because now you can share it with others and there is a good chance
they will be able to follow along. This is all well and good until you want to
do some number crunching. For some (most) problems a computer based solution
is orders of magnitude more effective.

For some once we enter the computational realm things get complex and unclear.
There are command lines, compilers, libraries, dependencies, formats and many
other scary things. Try asking someone working with scientific computations
how they solved their problem. More often than not you will get a very
technical answer to a physical problem. What is worse, try asking them to
\emph{show} you how they solved the problem. They tend to give you some
confusing source code and maybe some graphs. Even solving simple problems can be uninformative.

Next is the often \emph{steep} learning curve associated with computational
tools. There are some amazing tools out there but sometimes you just don't
feel like learning another gigantic peice of software from scratch when all
you wan't is some simple computations. Wouldn't it be nice to work using a
medium that is clear, elegant and powerful. An enviroment where your
computational solution is contained and presentable "as is" just like a pen
and paper solution.

Recently there has been a push to create systems where computer based
solutions can be written, executed and presented in a format that looks and
\emph{feels} like the traditional pen and paper approach to problem solving.
You may be familiar with some systems for creating "interactive digital
notebooks" or "computable documents". Mathematica provide a very popular
solution which exposes the mathematica language in a visually pleasing format.

\begin{quote}
\textit{ Mathematica notebooks are structured interactive documents that can
contain text, graphics, sound, calculations, typeset expressions, and user
interface elements.}
\end{quote}

More recently a solution has been developed for the python programming
language. The technology is called IPython notebooks.


\end{multicols}

\end{document}
