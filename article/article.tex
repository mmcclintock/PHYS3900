% latex template for a article
% written by Michael McClintock

\RequirePackage[l2tabu, orthodox]{nag} % warnings for obsolete stuff
\documentclass[10pt, a4paper]{article}  % article class

% font setup
\usepackage{ifluatex}
\ifluatex
    \usepackage[no-math]{fontspec} % keep standard math
    \setmainfont[Ligatures=TeX]{Linux Libertine}
    \setsansfont[Ligatures=TeX]{Linux Libertine}
    \setmonofont[Ligatures=TeX]{Inconsolata}
\else
    \usepackage[T1]{fontenc}
    \usepackage{inconsolata}
\fi
\usepackage{microtype}

% useful packages
\usepackage[svgnames]{xcolor}          % color commands
\usepackage{graphicx}                  % including graphics (use eps)
\usepackage{multicol}
\usepackage[margin=15mm]{geometry}     % set margins
\usepackage{amsmath,amssymb,cancel,bm} % math \bm for bold math
\usepackage[hidelinks]{hyperref}       % clickable links
\usepackage{url}                       % format urls
\usepackage{cleveref}                  % convenient refrencing use \Cref
\usepackage{paralist}                  % compact itemize/enumerate
\usepackage[hang, small, bf, margin=20pt]{caption} % captions for figures
\usepackage{setspace}%\onehalfspacing               % 1.5 line spacing
\setlength{\parindent}{0cm}\usepackage{parskip}    % paragraph skip
\usepackage[                                   % SI units
    load-configurations=abbreviations,         % unit abrev.
    separate-uncertainty=true,                 % plus minus in uncertainty
    inter-unit-product=\ensuremath{{}\cdot{}}, % dot between units
    per-mode=symbol-or-fraction                % per style
]{siunitx}
% \usepackage{todonotes} % use \todo{blah blah} options inline/noline/nolist/color
% \usepackage{minted}  % code highlighting requires pygments
% \usepackage[english]{babel} % other languages
% \usepackage[nottoc,numbib]{tocbibind} % refs in toc

\begin{document}

\begin{center}
    \rule[0.5ex]{1\columnwidth}{1pt}
    \\[4mm]
    \textbf{\textsc{\Huge Digital Notebooks for Computation}}
    \\[6mm]
    \textit{\Large A classy interface for scientific computing}
    \\[6mm]
    \textsc{\large By Michael McClintock}
    \\[4mm]
    \rule[0.5ex]{1\columnwidth}{1pt}
\end{center}

\begin{multicols}{2}

Some people are really good at solving problems with pen and paper. There is
just something special about solving a problem with just your mind, a pen and
some paper. When you're done there is a feeling of value associated with your
pages of notes and derivations. If you write it clearly with comments well all
the better because now you can share it with others and there is a good chance
they will be able to follow. As you know though this all falls apart when you
want to do some data processing on a computer.

For some once we enter the computational realm things get complex and unclear.




\end{multicols}

\end{document}
