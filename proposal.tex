% latex template for a report
% written by Michael McClintock

\RequirePackage[l2tabu, orthodox]{nag} % warnings for obsolete stuff
\documentclass[12pt,a4paper]{article}  % article class

% font setup
\usepackage{ifluatex}
\ifluatex
    \usepackage[no-math]{fontspec} % keep standard math
    \setmainfont[Ligatures=TeX]{Linux Libertine}
    \setsansfont[Ligatures=TeX]{Linux Biolinum}
    \setmonofont[Ligatures=TeX]{Inconsolata}
\else
    \usepackage[T1]{fontenc}
    \usepackage{inconsolata}
\fi
\usepackage{microtype}

% useful packages
\usepackage[svgnames]{xcolor}          % color commands
\usepackage{graphicx}                  % including graphics (use eps)
\usepackage[margin=20mm]{geometry}     % set margins
\usepackage{amsmath,amssymb,cancel,bm} % math \bm for bold math
\usepackage[hidelinks]{hyperref}       % clickable links
\usepackage{url}                       % format urls
\usepackage{cleveref}                  % convenient refrencing use \Cref
\usepackage{paralist}                  % compact itemize/enumerate
\usepackage[hang, small, bf, margin=20pt]{caption} % captions for figures
\usepackage{setspace}%\onehalfspacing               % 1.5 line spacing
\setlength{\parindent}{0cm}\usepackage{parskip}    % paragraph skip
\usepackage[                                   % SI units
    load-configurations=abbreviations,         % unit abrev.
    separate-uncertainty=true,                 % plus minus in uncertainty
    inter-unit-product=\ensuremath{{}\cdot{}}, % dot between units
    per-mode=symbol-or-fraction                % per style
]{siunitx}
% \usepackage{todonotes} % use \todo{blah blah} options inline/noline/nolist/color
% \usepackage{minted}  % code highlighting requires pygments
% \usepackage[english]{babel} % other languages
% \usepackage[nottoc,numbib]{tocbibind} % refs in toc

\begin{document}

\begin{center}
  \textsc{\Large Project Proposal: Five Minute Physics}
  \\[8mm]
  \textbf{Michael McClintock}, Alexander Van Nunen
\end{center}

\section*{Introduction and Background}

It is no secret that smartphones, tablets and other portable
multimedia devices are on the rise. As reported in February 2012,
487.7 million smartphones where shipped worldwide 2011.  This figure,
up 63\% from 2010, also meant that smarthphone sales have overtaken
the sale of personal computers \cite{canalys}. 

There are a number of factors that make smartphone based learning
(often referred to as ubiquitous learning or just ``u-learning") an
effective, if not inevitable, compliment to the more traditional
methods of course and content delivery. These factors include the
widespread smartphone reliance of university students and staff, high
convenience, powerful multimedia capabilities and good wireless and
internet connectivity \cite{worry}.

While many universities have developed campus wide tools for
smartphones. The use of u-learning solutions tailor-made for specific
courses or subject areas is not widely accepted \cite{procsmart}.
One reason is the lack of mature technology for content delivery that
addresses technical issues such as platform independence. Another is the
lack of examples and trials related to the design of content for
u-learning systems.

This project has two separate goals. The first is to find a simple
solution that allows the delivery of u-learning content to smartphones
and tablets. The second and more important goal is to produce and test
some u-learning material for first year physics students.

While subject specific u-learning material is a relatively new concept
there is a great deal of research into what makes good ``e-learning"
material where e-learning refers to the use of computers and internet
in learning. Many of the generally accepted approaches to e-learning
may be equally effective when transferred to a smartphone based
system.

\section*{Significance of Research and Expected Outcomes}

Any method or aid that helps students absorb and understand a topic at
a fundamental level is valuable to instructors. Finding this type of
tool for physics is even more significant because the ideas and
concepts of physics are generally more abstract and fundamental
compared to other subject areas. 

Designing, implementing and testing a u-learning system for a small
part of a first year physics course allows for evaluation of the
effectiveness of smartphone based systems. Following this project
instructors should have some evidence of the advantages or
disadvantages of u-learning systems for first year physics. As well as
some ideas about how to create their own systems.

\section*{Methodology and Design}

Again when it comes to designing the solution there are two separate
aspects. The first is the method of delivering the content and the
second is the design of the content itself. While the second aspect is
the main focus of the project it cannot be tested without an effective
solution to the problem of content delivery.


\subsubsection*{Content Delivery}

Web 2.0 technologies such as blogs, wikis and interactive websites
have changed the face of online interaction and collaboration
\cite{procsmart}. These online applications are becoming good enough
that at times its hard to know you're not running a native application
on your PC. Also it is worth noting that web technologies (HTML, CSS
and javascript) are open source, cross-platform, widespread and mature. 

The major problem of delivering content effectively to the large
number of smartphone and tablet like devices is dealing with their
differences.  To develop a native application for Android users, the
developer must be proficient with java and understand the Android
specific libraries. A similar but conflicting approach is needed for
iOS devices where one must understand Objective-C programming and the
Apple specific toolkits. This all becomes very overwhelming when the
goal is to achieve a simple content delivery system. 

Luckily web browsers exist on all devices that comply to standard web
technologies. This makes it possible to design u-learning content and
then deliver it to the multitude of device in a consistent
maintainable format. One project that aims to simplify this process is
called PhoneGap \cite{phonegap}. To achieve simple content delivery
this project aims to follow the approach of developing u-learning
material using web technologies that are deliverable to a number of
devices using technologies such as PhoneGap.

\subsubsection*{U-Learning Physics Content}

The content itself will be tailored to first year physics students at
the University of Queensland. More specifically this project will be
targeting the Fluids component of PHYS1171. The material will focus
on the topics of pressure/depth, buoyancy and the fluid dynamics of
non-viscous fluids. 

The approach to designing content will be to look at effective content
in e-learning systems aimed specifically at physics. A good example is
the design of effective physics videos \cite{vid}. Videos and
animation based content isn't the only way to keep users interested.
Simple things like the format and style of accompanying text and
graphics matter and these aspects will also be investigated.  If time
permits the possibility of adding interactive components like the
ability for users to leave comments will be looked at.

\section*{Timeline}

\begin{itemize}
\item Week 1-3. Familiarisation with topic. Produce Written proposal.  
\item Week 3-4. Research into physics content best
  practices.  Develop the structure of the content to be presented
\item Week 4-5. Develop preliminary content using web technologies
\item Week 5-6. Find/implement a simple method for delivering the
  preliminary content to physical devices 
\item Week 7-8. Refine material and prepare for trial on students 
\item week 8-9. Test system on students and collect feedback 
\item week 9-10. Analyse results and prepare report
\end{itemize}

\bibliographystyle{ieeetr}
\bibliography{refs}

\end{document}

% v: set spell:
